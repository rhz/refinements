\documentclass[a4paper,12pt]{article}

\usepackage[utf8]{inputenc}
\usepackage[T1]{fontenc}
\usepackage[british]{babel}
\usepackage{microtype}
\usepackage[sc]{mathpazo}
\usepackage{xspace}
\usepackage{enumitem}
\usepackage[usenames,dvipsnames,svgnames,table]{xcolor}
\usepackage{mathcommon}
\usepackage{kappa}

\newcommand{\ie}{i.e.\xspace}
\newcommand{\eg}{e.g.\xspace}
\newcommand{\lem}[1]{Lemma~\ref{lem:#1}}
\newcommand{\diagram}[1]{diagram~\ref{eq:#1}}
\newcommand{\anon}[1]{\left|#1\right|}
\newcommand{\gp}{\Gamma}
\newcommand{\id}{\vec{1}}
\newcommand{\rSGe}{\mathbf{rSGe}}
\newcommand{\shapes}{\mathcal{P}}
\newcommand{\edges}{\mathcal{E}}
\newcommand{\comatch}[1]{#1^\star}
\newcommand{\inv}[1]{#1^\dagger}
\newcommand{\rfns}[1]{\overline{#1}}
\newcommand{\ord}{\omega}
\newcommand{\egp}{\gp^\shapes}
\newcommand{\ogp}{\gp_\ord}

\tikzstyle{onarrow}=[fill=White,inner sep=2pt]

\renewcommand{\baselinestretch}{1.33}
\frenchspacing

\title{Ordered refinements}

\begin{document}

\maketitle

\begin{abstract}
  We present an alternative view of refinements in Kappa.
  It is a more algorithmic view that can be implemented efficiently.
  We show how the thermodynamic growth policy
  can be rephrased in this setting,
  resulting in smaller refinements.
  However this approach breaks the cognitively pleasing symmetries
  that the previous setting generates.
  We propose that the smallest and symmetric set of refinements
  can be obtained by taking the intersection of all ordered refinements.
\end{abstract}

\section{Refinements}

An extension $\phi: g \to h$ is an epi in $\rSGe_C$.
We build an extension by adding sites, agents and links
in a certain order.
Given the extension $\phi: g \to h$,
it is enough to specify the (total) order $<$ on new sites
$X := \sites_{\anon{h}} \setminus \phi_\sites(\sites_{\anon{g}})$
to recontruct the order in which
sites, agents and links have been added.
Note that only those sites not in the image of $g$ are ordered.
Because sites can only be added after
the agent they belong to
or the site they are bound to,
we assume that orders are \emph{path-continuous}:
for some $n \in \NN$ and all $x \in X$,
there is a finite sequence of sites $(x_i)_{0 \leq i \leq n}$
such that i) $x_0 \in \phi(\sites_{\anon{g}})$,
ii) $x_n = x$,
iii) $x_i,x_{i+1}$ ($0 \leq i < n$)
are either connected ($(x_i,x_{i+1}) \in \edges_{\anon{h}}$)
or belong to the same agent ($\sitemap(x_i) = \sitemap(x_{i+1})$),
and iv) $x_i < x_{i+1}$ for all $0 \leq i < n$.
% https://en.wikipedia.org/wiki/Initial_segment
A subset $X$ of $Y$ is an \emph{initial segment} of $(Y,<)$
if, whenever $y < x$ for some $x \in X$ and $y \in Y$,
then $y \in X$.
An injection $f: X \to Y$ is \emph{rigid} % wrt $<$
if $f(X)$ is an initial segment of $(Y,<)$.
Given $(X,<)$ and $(X',<')$,
we write $< \,\sqsubseteq_f\, <'$ if
$f$ is an injection of $X$ into $X'$
that preserves the order
($f(x) < f(y)$ if $x < y$ for all $x,y \in X$)
and is rigid.

We refer to the pair $(\phi,<)$
as an ordered extension.
To each extension $\phi: g \to h$
we will assign an order $\ord(\phi)$,
which we also write as $<_\phi$ sometimes.
% and only use this order to determine the site requests at $h$.
The assignment $\ord$ must be \emph{consistent}
in the sense that further extensions of $\phi$
must preserve the order $<_\phi$,
that is, for all $\phi'$ such that $\psi\,\phi = \phi'$
for some $\psi$, the function $\psi_\sites$ mapping the sites
in the image of $\phi$ to those in the image of $\phi'$
must be order-preserving.
An ordered extension $(\phi: g \to h, <_\phi)$
is further extended by an embedding $\psi: h \to h'$
if $<_\phi \,\sqsubseteq_{\psi_\sites}\, <_{\phi'}$
with $\phi' = \psi\,\phi$.

% TODO: choose one assigment of orders and say we are only
% interested in the embeddings that are rigid with respect to the order
% take subcategory of rigid embeddings

An ordered growth policy $\ogp$ is a function
that takes an ordered extension $(\phi: g \to h, <_\phi)$
and returns a map from agents $u$ in $h$
to a subset of the sites it can define
according to the contact graph,
\ie $\sitemap^{-1}_C(h_\agents(u))$.
We classify pairs $(\phi: g \to h, <_\phi)$ according to $\ogp$ as usual
\begin{enumerate}[label={(\roman*)}]
\item \emph{growing} if for all agents $u$ in $\agents_{\anon{h}}$,
  $h_\sites(\sitemap^{-1}_{\anon{h}}(u)) \subseteq \ogp(\phi, <_\phi)$,
\item \emph{mature} if for all agents $u$
  the inclusion is an equality, and
\item \emph{overgrown} otherwise.
\end{enumerate}
$\ogp$ must satisfy,
% for all ordered extensions $(\phi', <')$
% that further extend $(\phi, <)$,
% with $\psi \, \phi = \phi'$,
for all ordered extensions $(\phi, <)$, $(\phi', <')$
such that $\psi \, \phi = \phi'$ for some $\psi$
and $< \,\sqsubseteq_{\psi_\sites}\, <'$,
the \emph{local faithfulness} property,
$\ogp(\phi, <) = \ogp(\phi', {<'}) \, \psi_\agents$.
In words, this property mandates that site requests
are carried over when we further extend pairs $(\phi, <)$.
% ordered extensions.
%
% TODO: integrate the following paragraph
% Different orders can request different sites
% for the same extension $\phi$.
% We would like to find those orders
% that request the least amount of sites.
% Unfortunately we would have to
% try all possible orders to find the minimum,
% so instead we will define growth policies
% for which any order is good enough.
%
We write $\rfns{\ogp}$ for the set of mature extensions of $g$
given by the growth policy $\ogp$. % and order assigment $\omega$.
% and the order $<_\phi$ on extensions $\phi$ of $g$.
We will prove that
for any consistent assignment of orders $\omega$, % to extensions,
the set of mature extensions $\rfns{\ogp} = \set{\phi_i: g \to h_i}$
% determined by a growth policy
\emph{uniquely decomposes} $g$,
\ie for every mixture $m$ and embedding $\psi: g \to m$,
there exists a unique $i$ and $\psi'$
such that $\psi = \psi' \, \phi_i$.

To have a non-empty set of refinements,
we ask $\ogp$ to at least contain the sites already in $g$,
that is, for all agents $u$ in $\agents_{\anon{g}}$,
$g_\sites(\sitemap_{\anon{g}}^{-1}(u)) \subseteq
\ogp(\id_g, \varnothing)(u)$
with $\id_g$ the identity on $g$.
If $g_\sites(\sitemap_{\anon{g}}^{-1}(u)) =
\ogp(\id_g, \varnothing)(u)$,
the identity $\id_g$ is mature
and every other extension is overgrown.

\begin{lemma}
  \label{lem:inj}
  Let $g$ and $s$ be contact maps over $C$,
  $\ord$ a consistent assignment of orders to extensions of $g$
  and $\ogp$ a growth policy for $g$.
  Given two mature extensions
  $\phi_1: g \to h_1$, $\phi_2: g \to h_2$ in $\rfns{\ogp}$
  and a cospan $\gamma_1: h_1 \to s \gets h_2 :\gamma_2$
  such that $\gamma_1\,\phi_1 = \gamma_2\,\phi_2$,
  then $\phi_1 = \phi_2$.
  % then the mediating arrow $\phi$ of
  % the pullback of $\gamma_1,\gamma_2$ is an epi
  % and $\ord(\phi_1) \sqsupseteq \ord(\phi)
  % \sqsubseteq \ord(\phi_2)$.
  \begin{center}
    \begin{tikzpicture}[every node/.style={outer sep=3pt}]
      \node (g) at (3,5.5) {$g$};
      \node (h'') at (3,3.9) {$h''$};
      \node[anchor=west,yshift=-1pt] at (h'') {$\ni u$};
      \node (h') at (3,2.45) {$h'$};
      \node (h) at (3,1) {$h$};
      \node (h1) at (0,0) {$h_1$};
      \node (h2) at (6,0) {$h_2$};
      \node (s) at (3,-1) {$s$};
      \draw (g) edge[hom,out=200,in=90]
      node[above left] {$\phi_1$} (h1);
      \draw (g) edge[hom,out=-20,in=90]
      node[above right] {$\phi_2$} (h2);
      \draw (g) edge[hom,dashed,bend right=35]
      node[left] {$\phi$} (h);
      \draw[hom] (h) -- node[above left,pos=.4] {$\pi_1$} (h1);
      \draw[hom] (h) -- node[above right,pos=.4] {$\pi_2$} (h2);
      \draw[hom] (h1) -- node[below left,pos=.6] {$\gamma_1$} (s);
      \draw[hom] (h2) -- node[below right,pos=.6] {$\gamma_2$} (s);
      \draw[hom] (g) -- node[right] {$\phi''$} (h'');
      \draw[hom] (h'') -- node[right,pos=.45] {$\psi'$} (h');
      \draw[hom] (h') -- node[right,pos=.45] {$\psi$} (h);
    \end{tikzpicture}
  \end{center}
\end{lemma}
\begin{proof}
  Take the connected closure of the image of $\phi$,
  call it $h'$, and restrict $\phi$ to $h'$ to obtain
  the epi $\phi': g \to h'$ with $\phi = \psi\,\phi'$.
  Let $\pi_i' = \pi_i\,\psi$ with $i \in \set{1,2}$.
  Take the maximal subset $X_i$ of $\sites_{\anon{h'}}$
  such that $\pi_{i,\sites}'(X_i)$ is an initial segment
  of $<_i := <_{\phi_i}$.
  Remove from $h'$ all sites $x$ not in $X_i$
  and then all agents without any sites left.
  Call the result $h''$.
  % NOTE: the following is not true
  % Importantly, no site removed from $h'$ in this way
  % is bound to another site in $h'$
  % due to path-continuity of orders
  % and maximality of $X_i$.
  Due to path-continuity of orders
  every agent left in $h''$ is connected
  (perhaps indirectly) to the image of $g$.
  Thus the restriction of $\phi'$ to $h''$
  gives us an epi $\phi'': g \to h''$
  with $\phi' = \psi'\,\phi''$.
  Let $\pi_i'' = \pi_i\,\psi\,\psi'$.
  We have $<_{\phi''} =: {<''} \,\sqsubseteq\, <_i$
  since, by consistency,
  $\pi_{i,\sites}''$ are order-preserving
  and, by construction,
  $\pi_{i,\sites}''(\sites_{\anon{h''}})$
  is an initial segment of $<_i$.

  Assume $\pi_i$ are not isos,
  otherwise $\phi = \phi_i$.
  Then there must be an agent $u$ in $h''$
  whose image along $\pi_{1,\agents}''$
  has a site $x_1$ bound to some site in
  $\sites_{\anon{h_1}} \setminus \pi_{1,\sites}(\sites_{\anon{h}})$
  as $\phi_1$ is an epi.
  It must be that $h_{1,\sites}(x_1)$ is in
  $\ogp(\phi_1,{<_1})(\pi_{1,\agents}''(u))$
  since $\phi_1$ is mature.
  As shown above, ${<''} \sqsubseteq\, <_i$,
  % By consistency,
  % $\pi_{1,\sites}$ and $\pi_{2,\sites}$ are order-preserving.
  % If, in addition, we have that $\sites_{\anon{h'}}$
  % is minimal with respect to $\ord_1$ and $\ord_2$,
  % then $\ord_1 \sqsupseteq \ord' \sqsubseteq \ord_2$.
  % as $\sites_{\anon{h'}}$ is minimal
  % with respect to $\ord_1$ and $\ord_2$.
  so, by local faithfulness, we have that
  $h_{1,\sites}(x_1)$ must be in $\ogp(\phi'',{<''})(u)$
  and thus also in $\ogp(\phi_2,{<_2})(\pi_{2,\agents}''(u))$.
  However,
  % $\pi_{2,\agents}'(u)$ does not contain
  the image of $u$ along $\pi_{2,\agents}''$ does not contain
  a site of type $h_{1,\sites}(x_1)$
  as $x_1$ does not have a preimage via $\pi_{1,\agents}$.
  Hence, $\phi_2$ is not mature
  and we have found a contradiction.
  % Since $\phi_2$ is mature,
  % $\pi_{2,\agents}''(u)$ must have a site of type $h_{1,\sites}(x_1)$.
  %
  % Since $\phi$ is then an epi,
  % it must be that $\phi' = \phi$
  % as taking the connected closure of the image of an epi
  % % gives us back the whole codomain of $\phi$.
  % gives us back the codomain of $\phi$ intact.
\end{proof}

\begin{theorem}
  Let $g$ be a contact map over $C$,
  $\ord$ a consistent assignment of orders to extensions of $g$
  and $\ogp$ a growth policy for $g$.
  For all mixtures $m$ and embeddings $\psi: g \to m$,
  there exists a unique $\phi_i \in \rfns{\ogp}$
  and $\psi'$ such that $\psi = \psi' \, \phi_i$.
\end{theorem}
\begin{proof}
  \emph{Injectivity}:
  We prove that given two decompositions of $\psi$
  as $\gamma_1\,\phi_1$ and $\gamma_2\,\phi_2$
  with $\phi_1,\phi_2 \in \rfns{\ogp}$,
  then $\phi_1 = \phi_2$.
  This has been proved in \lem{inj}.

  \emph{Surjectivity}:
  Take any embedding $\psi$ of $g$ into a mixture $m$.
  We can restrict the codomain of $\psi$ to be
  the connected closure $n$ of the image of $\psi$ in $m$,
  resulting in an epi $\psi': g \to n$.
  Let us further restrict $n$ by removing
  (i) all sites not requested by the growth policy and
  % if it's a free site no problem
  % if it's bound, path-continuity
  (ii) all agents that have no sites requested by the growth policy.
  Call the result $o$.
  It has the same number of connected components as $g$
  since $\ogp$ can only request sites
  for which there is a path to sites in the image of $\psi$.
  We thus obtain an epi $\phi: g \to o$
  which is mature with respect to $\ogp$ since,
  by construction, its image contains all sites
  requested by $\ogp$ and no other foreign site.
  It is easy to see that $\phi$ factorises $\psi$.
\end{proof}


\section{Energy-based refinements}

Given contact graph $C$,
a set $\shapes$ of energy patterns (contact maps over $C$),
a rule $r = (r_L,r_R)$,
and a consistent assignment of orders $\ord$
to extensions of $r_L$,
we define our growth policy $\gp_\ord$ for $r_L$ as follows.
Suppose $\phi: r_L \to g$ is an extension of $r_L$.
To simplify notation,
we will write $\gp_\ord(\phi)$ for $\gp_\ord(\phi, <_\phi)$.
We set $\gp(\phi)$ to request
a site $z$ in $\sitemap_C^{-1}(g_\agents(u))$
at agent $u$ in $\agents_{\anon{g}}$ iff either
\begin{enumerate}[label={(\roman*)}]
\item $u = \phi_\agents(u_0)$ and $z = \phi_\sites(z_0)$
  for some $u_0$ in $\agents_{\anon{r_L}}$ and
  $z_0$ in $\sites_{\anon{r_L}}$; or
% TODO: glue only at the beginning and at each extension
% push forward the glueing to \phi_1
% (there's a unique push-forward if there is one?
% sometimes there's no push-forward because we excluded
% that minimal glueing in the extension)
\item $\phi$ factorises as $\phi_2\,\phi_1$
  % with $\phi_1: r_L \to g_1$, $\phi_2: g_1 \to g$
  such that $\ord(\phi_1) \sqsubseteq \ord(\phi)$
  and there is a relevant minimal glueing
  $\gamma: p \to s \gets g_1 :\theta$
  with $p$ in $\shapes$
  and some agent $u_1$ in $\agents_{\anon{g_1}}$
  and site $z_1$ in $\sitemap_{\anon{s}}^{-1}(\theta_\agents(u_1))$
  such that $u = \phi_{2,\agents}(u_1)$ and $z = s_\sites(z_1)$,
  as shown in \diagram{energy-gp}; or
  \begin{equation}
    \label{eq:energy-gp}
    \tikz[baseline=-2.5,thick]{
      \node (p) at (0,0) {$p$};
      \node (s) at (2,-1.2) {$s$};
      \node (l) at (4,1.8) {$r_L$};
      \node (g1) at (4,0) {$g_1$};
      \node[anchor=east] at (g1) {$u_1 \!\in{}\,$};
      \node (g) at (6,-1.2) {$g$};
      \node[anchor=west] at (g) {${}\ni u$};
      \draw[hom] (l) -- node[pos=.45,left] {$\phi_1$} (g1);
      \draw[hom] (p) -- node[below left] {$\gamma$} (s);
      \draw[hom] (g1) -- node[below right] {$\theta$} (s);
      \draw[hom] (g1) -- node[below left] {$\phi_2$} (g);
      \draw (l) edge[hom,bend left=30] node[above right] {$\phi$} (g);
    }
  \end{equation}
\item $z = g_\sites(z_2)$ for some $z_2$ in $\sites_{\anon{g}}$
  such that $z_2 \,\edges_{\anon{g}}\, z_3$
  and $g_\sites(z_3)$ in $\gp(\phi)(u)$.
\end{enumerate}
In words, clause (i) ensures
that all sites in $r_L$ are requested by the growth policy
while clause (ii) adds sites $z$ in $\sites_C$
corresponding to sites $z_1$ in $\sites_{\anon{s}}$
which appear by glueing with $p$
at some point between $r_L$ and $g$.
Clause (iii), on the other hand,
asks for sites that are bound to sites
that are requested by the growth policy
so that extensions that avoid minimal glueings are not overgrown.

We refer to the extension $\phi_2: g_1 \to g$
as a \emph{rewind} of $\phi$
whenever $\ord(\phi_1) \sqsubseteq \ord(\phi)$
and say that the request of $z$ at $u$ originates from $u_1$.
By rewinding extensions we can remember
which sites have been asked for in the past.
Symmetrically, we define a growth policy $\comatch{\gp}$ for $r_R$
by applying the same definition to the inverse rule $\inv{r}$.
Finally, we define our growth policy $\egp$
as the union of both growth policies,
that is, $\gp^\shapes(\phi)(u) = \gp(\phi)(u)
\,\cup\, \comatch{\gp}(\comatch{\phi})(u)$.

\begin{theorem}
  Let $r = (r_L,r_R)$ be a rule
  and $\ord$ a consistent assignment of orders
  to extensions of $r_L$.
  The above $\gp^\shapes$ is indeed a growth policy for $r_L$
  and the refined family of rules
  induced by the set of mature extensions $\overline{\gp^\shapes}$
  is $\shapes$-balanced and finite.
\end{theorem}
\begin{proof}
  \emph{Growth policy}:
  We have to prove that
  $\gp^\shapes$ fulfils the local faithfulness property.
  Using the notation in \diagram{energy-gp},
  we see that $\gp^\shapes(\phi_1)(u_1) \subseteq \gp^\shapes(\phi)(u)$
  as every request for a site in $g_1$ will propagate to $g$
  by definition.
  To prove the other direction
  amounts to verify that the requests generated by rewinds
  do not depend on the choice of factorisation
  as $\gp^\shapes(\phi)(u)$ must be a subset of
  $\gp^\shapes(\phi_1)(u_1)$ for every $\phi_1$
  with $\omega(\phi_1) \sqsubseteq \omega(\phi)$.
  % So, assume there is an alternative factorisation of $\phi$
  % through $g_2$ with $\phi = \psi_1\,\psi_2$
  % and $\omega(\psi_1) \sqsubseteq \omega(\phi)$
  % which gives rise to a site request in $u$
  % originating from some $u_2$ in $g_2$,
  % as in the following diagram.
  So, assume there are two factorisations of $\phi$
  given by $\phi_2\,\phi_1 = \phi = \psi_2\,\psi_1$
  with $\omega(\phi_1) \sqsubseteq \omega(\phi)
  \sqsupseteq \omega(\psi_1)$
  and consider a site request in $u$
  originating from some $u_2$ in $g_2$,
  as in the following diagram.
  \begin{center}
    \begin{tikzpicture}
      \node (l) at (3,2.2) {$r_L$};
      \node (g0) at (3,1) {$g_0$};
      \node (g1) at (0,0) {$g_1$};
      \node[anchor=east,xshift=-5pt] at (g1) {$u_1 \in$};
      \node (g2) at (6,0) {$g_2$};
      \node[anchor=west,xshift=5pt] at (g2) {$\ni u_2$};
      \node (g) at (3,-1) {$u \in g$};
      \draw (l) edge[hom,bend right=20] node[above left] {$\phi_1$} (g1);
      \draw (l) edge[hom,bend left=20] node[above right] {$\psi_1$} (g2);
      \draw[hom] (g0) -- (g1);
      \draw[hom] (g0) -- (g2);
      \draw[hom] (g1) -- node[below left] {$\phi_2$} (g);
      \draw[hom] (g2) -- node[below right] {$\psi_2$} (g);
      \draw[hom,dotted] (l) -- (g0);
    \end{tikzpicture}
  \end{center}
  % % Since $\omega(\phi_1) \sqsubseteq \omega(\phi)
  % % \sqsupseteq \omega(\psi_1)$,
  % Since the images of the sites in $g_1$ and $g_2$
  % are minimal with respect to those in $g$,
  % the sites in $g_1$ must form a subset of those in $g_2$ or vice versa.
  % Say $g_{1,\sites}(\sites_{\anon{g_1}}) \subseteq
  % g_{2,\sites}(\sites_{\anon{g_2}})$.
  (The argument really is the same as in non-ordered refinements,
  we just need to check the orders are OK
  but they are by definition ...)
\end{proof}


\end{document}

%%% Local Variables:
%%% mode: latex
%%% TeX-master: t
%%% End:
